\documentclass[12pt]{article}
\usepackage[utf8]{inputenc}
\usepackage[swedish]{babel}
\usepackage{graphicx}
\usepackage{amsmath}
\usepackage{subfigure}
\usepackage{fullpage}
\usepackage{hyperref}
\usepackage[T1]{fontenc}


\setlength{\parindent}{0pt}
\setlength{\parskip}{0.5cm}

\newcommand{\stadga}{
\newcounter{a}
\setcounter{a}{0}
\newcounter{b}[a]
\setcounter{b}{0}
\newcounter{c}[b]
\setcounter{c}{0}
}

\newcommand{\kapitel}[1]{
\addtocounter{a}{1}
\newpage
\begin{center}
\phantomsection
\huge
Kapitel \arabic{a}. #1
\addcontentsline{toc}{section}{Kapitel \arabic{a}. #1} \\
\end{center}
\setcounter{b}{0}
\setcounter{c}{0}
}

\newcommand{\paragraf}[1]{
\addtocounter{b}{1}
\phantomsection
\Large
\hangindent0em
§ \arabic{a}.\arabic{b}. #1
\addcontentsline{toc}{subsection}{§ \arabic{a}.\arabic{b}. #1} \\
\setcounter{c}{0}
}

\newcommand{\underparagraf}[1]{
\addtocounter{c}{1}
\large
\hangindent2em
§ \arabic{a}.\arabic{b}.\arabic{c} #1
\\ \\
}


\begin{document}
\stadga
\begin{center}
\Huge
Reglemente för F-spexet\\
Org. nr: 802429-6561
\end{center}
\normalsize
Originalet skrivet sommaren 2020; första upplagan 2020-10-01\\
Baserad på Fysikteknologsektionens stadga och reglemente från 2002\\
Beatriz Bento Hansson, Ebba Grönfors, Anna Romeborn\\
 \\
Uppdaterad 2022-08-03 av Oskar Vallhagen\\
\\
Uppdaterad 2023-06-18 av Johan Ödesjö\\
\\
Uppdaterad 2023-09-12 av Johan Ödesjö\\
\\
% Uppdaterad 20??-??-?? Förnamn Efternamn\\
% \\
\newpage 
\tableofcontents
\kapitel{Allmänt}
\kapitel{Medlemmar}
\kapitel{Spexmötet}
\kapitel{Kollegium, valberedning och personval}

\paragraf{Invalsperiod}
\underparagraf{Varje kollegium ansvarar för inval under en invalsperiod som pågår från inval av kollegiet fram till en kalendervecka efter att de har meddelat Styret om sitt beslut.}
\underparagraf{Efter invalsperioden kan poster fyllnadsväljas av Styret. De behöver inte explicit ha vakantsatts för detta.}

\paragraf{Valberedningen}
\underparagraf{Valberedningen nominerar kandidater till poster enligt stadgarna.}
\underparagraf{En medlem i valberedningen kan inte nomineras till en styrelse- eller förtroendepost.}

\paragraf{Manuskollegiet}
\underparagraf{Manuskollegiet väljer in en manusgrupp för det kommande verksamhetsåret minst fyra månader före verksamhetsårets start.}
\underparagraf{En medlem i manuskollegiet kan inte väljas in till manusgruppen under invalsperioden.}

\paragraf{Skådespelarkollegiet}
\underparagraf{Skådespelarkollegiet väljer in medlemmar till ensemblen nära verksamhetsårets början. }
\underparagraf{En medlem i skådespelarkollegiet kan inte väljas in till ensemblen under invalsperioden.}
%Skådespelarkollegiet kan i samråd med spexiga kollegiet även välja in en regissör.

\paragraf{Orkesterkollegiet}
\underparagraf{Orkesterkollegiet väljer in medlemmar till orkestern nära verksamhetsårets början.}
%En medlem i orkesterkollegiet kan inte välja in sig själv till att spela ett specifikt instrument i orkestern om det medför att en annan sökande till samma instrument inte får posten.
\underparagraf{En medlem i orkesterkollegiet kan inte väljas in till orkestern under invalsperioden.}

\paragraf{Spexiga kollegiet}
\underparagraf{Spexiga kollegiet väljer in medlemmar till övriga spexgrupper nära verksamhetsårets början.}
% NÅGONTING (som är för kontroversiellt för den första versionen) SKA STÅ HÄR OM ATT VARA SNÄLL
% Något i stil med:
% Medlemmar i spexiga kollegiet som även söker en post som väljs av spexiga kollegiet ska i första hand ta hänsyn till övriga sökandes vilja.

%Spexiga kollegiet kan i samråd med skådespelarkollegiet även välja in en regissör.
\paragraf{Ändringar av kollegier}
\underparagraf{Om spexmötet bedömer det lämpligt kan de välja in en annan uppsättning kollegier än de som beskrivs ovan.}
\underparagraf{Spexmötet kan omfördela vilket kollegium som ansvarar för att välja in vilka poster.}
\underparagraf{Spexmötet kan ge kollegierna tillstånd att själva omfördela vilket kollegium som ansvarar för att välja in en specifik post.}

\paragraf{Personval}

\kapitel{Poster och spexgrupper}

%\paragraf{Förteckning över poster}

\paragraf{Styret}
\underparagraf{Styret består av styrelsen och förman tillsammans. Styrelsens poster definieras i stadgan.}
%\underparagraf{Ordförande}
%\underparagraf{Kassör}
%\underparagraf{Ledamöter}
%\paragraf{Förman}
%\underparagraf{Förmans ansvar är definierat i stadgan.}

\paragraf{Revisor}

\paragraf{Ensemblen}
\underparagraf{Ensemblen består av 6-7 personer. Spexmötet beslutar i samråd med Manusgruppen antalet personer i Ensemblen innan Skådespelarkollegiet väljs in.}

\paragraf{Koreograf}
\underparagraf{Koreograferna är 0-3 personer.}

\paragraf{Kostym}
\underparagraf{Kostymgruppen består av minst 2 personer.}

\paragraf{Kuplettsamordnare}
\underparagraf{Kuplettsamordnaren är 1-2 personer.}

\paragraf{Ljud och ljus}
\underparagraf{Ljud- och ljusgruppen består av 2-4 personer.}

\paragraf{Manus}
\underparagraf{Manusgruppen består av 2-5 personer.}

\paragraf{Mat}
\underparagraf{Matgruppen består av minst 3 personer.}

\paragraf{Media}
\underparagraf{Mediagruppen består av minst 3 personer.}

\paragraf{MåBra}
\underparagraf{MåBragruppen består av minst 2 personer.}

\paragraf{Orkestern}
\underparagraf{Orkestern består av 4-13 personer.}

\paragraf{Producent}
\underparagraf{Producenten är 1 person.}

\paragraf{Regissör}
\underparagraf{Regissörerna är 1-3 personer.}%, varav max 2 inte tillhör ensemblen.}

\paragraf{Scen}
\underparagraf{Scengruppen består av minst 3 personer.}

\paragraf{Smink}
\underparagraf{Sminkgruppen består av 1-3 personer.}

\paragraf{Tillfälliga poster}
\underparagraf{Om spexiga kollegiet bedömer det lämpligt kan de i samråd med Styret införa ytterligare poster för ett enskilt verksamhetsår.}
\underparagraf{Antalet medlemmar i en spexgrupp kan frångås om kollegiet bedömer det lämpligt i samråd med Styret.}

\paragraf{Dubbelpostande}
\underparagraf{En och samma medlem kan inneha flera olika poster i spexet under samma verksamhetsår.}

\kapitel{Ekonomi}

%\paragraf{Man får INTE vara en feffe.}

\kapitel{Uthyrningsregler}

%\paragraf{Porslin}
%\underparagraf{Spexets porslin kan hyras ut inom och utanför F-sektionen.}
%\underparagraf{Styret ansvarar för uthyrning av porslin.}

%\underparagraf{Inom sektionen kostar det 100 kr att hyra porslin till ett normalstort evenemang.}

%\underparagraf{Utanför sektionen kostar det 200 kr att hyra porslin till ett normalstort evenemang.}

%\underparagraf{Priset för uthyrning kan anpassas om evenemanget anses vara mycket stort eller mycket litet.}
%\paragraf{Mittenburen}

\paragraf{Utrustning och rekvisita}
\underparagraf{Styret ansvarar för uthyrning och utlåning av spexets utrustning och rekvisita.}
\underparagraf{Rekvisita som används till den rådande spexproduktionen hyrs eller lånas inte ut om inte alla föreställningar redan har varit. }

%Någon i Styret ska vara medveten om utlånet och besluta om ev. ersättning om något går sönder eller försvinner.


\end{document}
